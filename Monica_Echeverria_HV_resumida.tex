%!TEX TS-program = xelatex
%!TEX encoding = UTF-8 Unicode
% Awesome CV LaTeX Template for CV/Resume
%
% This template has been downloaded from:
% https://github.com/posquit0/Awesome-CV
%
% Author:
% Claud D. Park <posquit0.bj@gmail.com>
% http://www.posquit0.com
%
%
% Adapted to be an Rmarkdown template by Mitchell O'Hara-Wild
% 23 November 2018
%
% Template license:
% CC BY-SA 4.0 (https://creativecommons.org/licenses/by-sa/4.0/)
%
%-------------------------------------------------------------------------------
% CONFIGURATIONS
%-------------------------------------------------------------------------------
% A4 paper size by default, use 'letterpaper' for US letter
\documentclass[11pt,a4paper,]{awesome-cv}

% Configure page margins with geometry
\usepackage{geometry}
\geometry{left=1.4cm, top=.8cm, right=1.4cm, bottom=1.8cm, footskip=.5cm}


% Specify the location of the included fonts
\fontdir[fonts/]

% Color for highlights
% Awesome Colors: awesome-emerald, awesome-skyblue, awesome-red, awesome-pink, awesome-orange
%                 awesome-nephritis, awesome-concrete, awesome-darknight

\definecolor{awesome}{HTML}{00A388}

% Colors for text
% Uncomment if you would like to specify your own color
% \definecolor{darktext}{HTML}{414141}
% \definecolor{text}{HTML}{333333}
% \definecolor{graytext}{HTML}{5D5D5D}
% \definecolor{lighttext}{HTML}{999999}

% Set false if you don't want to highlight section with awesome color
\setbool{acvSectionColorHighlight}{true}

% If you would like to change the social information separator from a pipe (|) to something else
\renewcommand{\acvHeaderSocialSep}{\quad\textbar\quad}

\def\endfirstpage{\newpage}

%-------------------------------------------------------------------------------
%	PERSONAL INFORMATION
%	Comment any of the lines below if they are not required
%-------------------------------------------------------------------------------
% Available options: circle|rectangle,edge/noedge,left/right

\photo{images/Moni.jpg}
\name{Mónica}{Echeverría Burbano}


\mobile{(+57) 3212747177}
\email{\href{mailto:monicaeche@gmail.com}{\nolinkurl{monicaeche@gmail.com}}}
\orcid{0000-0002-6033-1085}

% \gitlab{gitlab-id}
% \stackoverflow{SO-id}{SO-name}
% \skype{skype-id}
% \reddit{reddit-id}

\quote{\href{https://scienti.minciencias.gov.co/cvlac/visualizador/generarCurriculoCv.do?cod_rh=0001636025}{\emph{CvLAC}}
• \href{https://mecheverria8.wixsite.com/monicaechefoto}{\emph{Book
fotográfico}}}

\usepackage{booktabs}

\providecommand{\tightlist}{%
	\setlength{\itemsep}{0pt}\setlength{\parskip}{0pt}}

%------------------------------------------------------------------------------



% Pandoc CSL macros
\newlength{\cslhangindent}
\setlength{\cslhangindent}{1.5em}
\newlength{\csllabelwidth}
\setlength{\csllabelwidth}{2em}
\newenvironment{CSLReferences}[3] % #1 hanging-ident, #2 entry spacing
 {% don't indent paragraphs
  \setlength{\parindent}{0pt}
  % turn on hanging indent if param 1 is 1
  \ifodd #1 \everypar{\setlength{\hangindent}{\cslhangindent}}\ignorespaces\fi
  % set entry spacing
  \ifnum #2 > 0
  \setlength{\parskip}{#2\baselineskip}
  \fi
 }%
 {}
\usepackage{calc}
\newcommand{\CSLBlock}[1]{#1\hfill\break}
\newcommand{\CSLLeftMargin}[1]{\parbox[t]{\csllabelwidth}{\honortitlestyle{#1}}}
\newcommand{\CSLRightInline}[1]{\parbox[t]{\linewidth - \csllabelwidth}{\honordatestyle{#1}}}
\newcommand{\CSLIndent}[1]{\hspace{\cslhangindent}#1}

\begin{document}

% Print the header with above personal informations
% Give optional argument to change alignment(C: center, L: left, R: right)
\makecvheader

% Print the footer with 3 arguments(<left>, <center>, <right>)
% Leave any of these blank if they are not needed
% 2019-02-14 Chris Umphlett - add flexibility to the document name in footer, rather than have it be static Curriculum Vitae
\makecvfooter
  {19 de noviembre de 2022}
    {Mónica Echeverría Burbano~~~·~~~Hoja de Vida Resumida}
  {\thepage}


%-------------------------------------------------------------------------------
%	CV/RESUME CONTENT
%	Each section is imported separately, open each file in turn to modify content
%------------------------------------------------------------------------------



\hypertarget{acerca-de-muxed}{%
\section{Acerca de mí}\label{acerca-de-muxed}}

Mi trabajo profesional ha estado vinculado con el respeto de los
Derechos Humanos, el acceso a la justicia, la resolución de conflictos y
a la equidad de género, desde la realización de estrategias
comunicativas participativas a favor del cambio social. Investigadora de
medios de comunicación,con experiencia en el análisis del quehacer
periodístico en diferentes contextos, la prevención de las violencias
basadas en género en las universidades y diferentes comunidades. Cuento
con experiencia en la estructuración pedagógica de procesos de
aprendizaje virtuales y presenciales

\hypertarget{investigaciuxf3n}{%
\section{Investigación}\label{investigaciuxf3n}}

\begin{cvskills}
  \cvskill
    {Líneas de Investigación}
    {Comunicación estratégica • Comunicación con enfoque de derechos humanos • Género • Ciclo de vida • \newline
    Análisis de medios}
\end{cvskills}

\hypertarget{educaciuxf3n}{%
\section{Educación}\label{educaciuxf3n}}

\begin{cventries}
    \cventry{Doctorado en Investigación de Medios de Comunicación}{Universidad Carlos III de Madrid}{Madrid, España}{2022}{}\vspace{-4.0mm}
    \cventry{Maestría en Igualdad de género en ámbito público y privado}{Universidad Jaime I}{Castellón de la Plana, España}{2017}{}\vspace{-4.0mm}
    \cventry{Máster en Derechos Fundamentales}{Universidad Carlos III de Madrid}{Madrid, España}{2011}{}\vspace{-4.0mm}
    \cventry{Especialización en Resolución de Conflictos}{Pontificia Universidad Javeriana}{Bogotá, Colombia}{2009}{}\vspace{-4.0mm}
    \cventry{Comunicación Social}{Universidad del Cauca}{Popayán, Colombia}{2006}{}\vspace{-4.0mm}
\end{cventries}

\hypertarget{experiencia-laboral-y-docente}{%
\section{Experiencia Laboral y
Docente}\label{experiencia-laboral-y-docente}}

Para una lista completa y descripción de responsabilidades, por favor
consulta mi
\href{https://github.com/JDLeongomez/Moni_HV/raw/main/Monica_Echeverria_HV.pdf}{Hoja
de Vida Académica}.

\begin{cventries}
    \cventry{Programa de Comunicación Social y periodismo}{Universidad Central}{Bogotá, Colombia}{2017 - Actualmente}{\begin{cvitems}
\item Gestión de la comunicación (4 horas semanales - 2018 - Actualmente)
\item Prácticas profesionales (4 horas semanales - 2022 - Actualmente)
\item Gestión de recursos (3 horas semanales - 2021)
\item Opción de grado Comunicación y DDHH - módulo Comunicación y Género (4 horas semanales - 2019 - 2021)
\item Dirección semillero de comunicación y DDHH (2 horas semanales  - 2017 - 2021)
\item Taller de acción social (4 horas semanales - 2017 - 2018)
\item Proyecto de línea - investigación (4 horas semanales - 2018)
\item Prácticas 1 (3 horas semanales - 2017 - 2018)
\end{cvitems}}
\end{cventries}

\hypertarget{logros}{%
\section{Logros}\label{logros}}

Para información sobre \textbf{becas} y \textbf{premios}, por favor
visita mi
\href{https://github.com/JDLeongomez/Moni_HV/raw/main/Monica_Echeverria_HV.pdf}{Hoja
de Vida Académica}.

\hypertarget{publicaciones}{%
\section{Publicaciones}\label{publicaciones}}

\hypertarget{section}{%
\subsection{\texorpdfstring{\textbf{Artículos}}{}}\label{section}}

\begingroup
\setlength{\parindent}{-0.5in}
\setlength{\leftskip}{0.5in}

\textbf{Echeverría-Burbano, M.} \& Walzer Moskovic, A. (en prensa). El
periodismo y las violencias sexuales en el conflicto armado colombiano.
\emph{Revista Opción}

\textbf{Echeverría-Burbano, M.}, \& Leguízamo-Serna, L.R. (2021). El
quehacer periodístico en casos de violencia sexual contra mujeres en
Colombia. \emph{Nómadas, 53}, 268-277.
\url{https://doi.org/10.30578/nomadas.n53a16}

\textbf{Echeverría-Burbano, M.} (2021). El Mito del Amor Romántico en el
relato periodístico sobre los feminicidios de mujeres mayores de edad en
Colombia. \emph{Comunicación Y Sociedad, 18}, 1-19.
\url{https://doi.org/10.32870/cys.v2021.7762}

\endgroup

\hypertarget{section-1}{%
\subsection{\texorpdfstring{\textbf{Capítulos de Libro}}{}}\label{section-1}}

\begingroup
\setlength{\parindent}{-0.5in}
\setlength{\leftskip}{0.5in}

\textbf{Echeverría-Burbano, M.}, \& Leguízamo-Serna, L.R. (2022). El
problema de poner a los victimarios en el centro de las historias de la
violencia en contra de las niñas y las mujeres. El caso del feminicidio
agravado de Yuliana Samboní. En L. Manrique Villanueva \& S.L. Ruiz
Moreno (Eds.), \emph{Mujeres, comunicación y cambio social} (pp.~31-52).
Ediciones USTA. \url{http://hdl.handle.net/11634/44849}

\textbf{Echeverría-Burbano, M.}, \& Baquero Gaitán, M. (2020). El
feminicidio en la prensa escrita colombiana. Un acercamiento desde el
análisis de contenido. En J.M. Pereira G. (Ed.), \emph{Comunicación,
información y lenguajes de la memoria} (pp.~1-17). Editorial Pontificia
Universidad Javeriana.
\url{https://www.javeriana.edu.co/unesco/comunicacioninformacion/contenido/ponencias/tema6/pdf/ponencia_03.pdf}

\textbf{Echeverría-Burbano, M.}, \& Niño Sandoval, Y.P. (2019). Redes
sociales y naturalización de la violencia contra las mujeres. En J.M.
Pereira G. (Ed.), \emph{Buen vivir, cuidado de la casa común y
reconciliación} (pp.~1-15). Editorial Pontificia Universidad Javeriana.
\url{https://javeriana.edu.co/unesco/buenvivir/contenido/ponencias/tema1/pdf/ponencia_15.pdf}

\textbf{Echeverría-Burbano, M.} (2019). Militares de EEUU y abusos
sexuales a niñas en Colombia, un análisis desde los medios colombianos.
En R. Cabral, A.I. Arévalo Salinas, G. Vilar Sastre \& T. Al Najjar
Trujillo (Eds.), \emph{Estudios interdiciplinarios:Paz y comunicación}
(pp.~70-80). Red de universidades internacionales y nacionales.

\endgroup

\hypertarget{roles-editoriales}{%
\section{Roles Editoriales}\label{roles-editoriales}}

\begin{cventries}
    \cventry{Journals Incluyen}{Par Ad Hoc}{Journals Nacionales e Internacionales}{Desde 2018}{\begin{cvitems}
\item Revista Via Iuris
\item Estudios sobre el Mensaje Periodístico
\end{cvitems}}
\end{cventries}

\hypertarget{organizaciuxf3n-de-eventos-acaduxe9micos}{%
\section{Organización de Eventos
Académicos}\label{organizaciuxf3n-de-eventos-acaduxe9micos}}

\begin{cventries}
    \cventry{Organizadora}{Respondiendo a las injusticias epistemicas}{Universidad Central}{2020}{}\vspace{-4.0mm}
    \cventry{Organizadora}{Para nosotras, pero con nosotras}{Universidad Central}{2020}{}\vspace{-4.0mm}
    \cventry{Organizadora}{Estigmas y reparaciones, retos para las victimas de violencia sexual en
conflictos armados}{Universidad Central}{2019}{}\vspace{-4.0mm}
    \cventry{Organizadora}{Proyecto Tumaco}{Universidad Central}{2018}{}\vspace{-4.0mm}
    \cventry{Organizadora}{Lanzamiento del observatorio de Género y No es Hora
de Calla}{Universidad Central}{2017}{}\vspace{-4.0mm}
    \cventry{Organizadora}{Hablemos de diversidad, género y nuevas tecnologias}{Universidad Central}{2017}{}\vspace{-4.0mm}
\end{cventries}



\end{document}
