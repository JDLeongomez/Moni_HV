%!TEX TS-program = xelatex
%!TEX encoding = UTF-8 Unicode
% Awesome CV LaTeX Template for CV/Resume
%
% This template has been downloaded from:
% https://github.com/posquit0/Awesome-CV
%
% Author:
% Claud D. Park <posquit0.bj@gmail.com>
% http://www.posquit0.com
%
%
% Adapted to be an Rmarkdown template by Mitchell O'Hara-Wild
% 23 November 2018
%
% Template license:
% CC BY-SA 4.0 (https://creativecommons.org/licenses/by-sa/4.0/)
%
%-------------------------------------------------------------------------------
% CONFIGURATIONS
%-------------------------------------------------------------------------------
% A4 paper size by default, use 'letterpaper' for US letter
\documentclass[11pt,a4paper,]{awesome-cv}

% Configure page margins with geometry
\usepackage{geometry}
\geometry{left=1.4cm, top=.8cm, right=1.4cm, bottom=1.8cm, footskip=.5cm}


% Specify the location of the included fonts
\fontdir[fonts/]

% Color for highlights
% Awesome Colors: awesome-emerald, awesome-skyblue, awesome-red, awesome-pink, awesome-orange
%                 awesome-nephritis, awesome-concrete, awesome-darknight

\definecolor{awesome}{HTML}{00A388}

% Colors for text
% Uncomment if you would like to specify your own color
% \definecolor{darktext}{HTML}{414141}
% \definecolor{text}{HTML}{333333}
% \definecolor{graytext}{HTML}{5D5D5D}
% \definecolor{lighttext}{HTML}{999999}

% Set false if you don't want to highlight section with awesome color
\setbool{acvSectionColorHighlight}{true}

% If you would like to change the social information separator from a pipe (|) to something else
\renewcommand{\acvHeaderSocialSep}{\quad\textbar\quad}

\def\endfirstpage{\newpage}

%-------------------------------------------------------------------------------
%	PERSONAL INFORMATION
%	Comment any of the lines below if they are not required
%-------------------------------------------------------------------------------
% Available options: circle|rectangle,edge/noedge,left/right

\photo{images/Moni.jpg}
\name{Mónica}{Echeverría Burbano}


\mobile{(+57) 3212747177}
\email{\href{mailto:monicaeche@gmail.com}{\nolinkurl{monicaeche@gmail.com}}}
\orcid{0000-0002-6033-1085}

% \gitlab{gitlab-id}
% \stackoverflow{SO-id}{SO-name}
% \skype{skype-id}
% \reddit{reddit-id}

\quote{\href{https://scienti.minciencias.gov.co/cvlac/visualizador/generarCurriculoCv.do?cod_rh=0001636025}{\emph{CvLAC}}
• \href{https://mecheverria8.wixsite.com/monicaechefoto}{\emph{Book
fotográfico}}}

\usepackage{booktabs}

\providecommand{\tightlist}{%
	\setlength{\itemsep}{0pt}\setlength{\parskip}{0pt}}

%------------------------------------------------------------------------------



% Pandoc CSL macros
\newlength{\cslhangindent}
\setlength{\cslhangindent}{1.5em}
\newlength{\csllabelwidth}
\setlength{\csllabelwidth}{2em}
\newenvironment{CSLReferences}[3] % #1 hanging-ident, #2 entry spacing
 {% don't indent paragraphs
  \setlength{\parindent}{0pt}
  % turn on hanging indent if param 1 is 1
  \ifodd #1 \everypar{\setlength{\hangindent}{\cslhangindent}}\ignorespaces\fi
  % set entry spacing
  \ifnum #2 > 0
  \setlength{\parskip}{#2\baselineskip}
  \fi
 }%
 {}
\usepackage{calc}
\newcommand{\CSLBlock}[1]{#1\hfill\break}
\newcommand{\CSLLeftMargin}[1]{\parbox[t]{\csllabelwidth}{\honortitlestyle{#1}}}
\newcommand{\CSLRightInline}[1]{\parbox[t]{\linewidth - \csllabelwidth}{\honordatestyle{#1}}}
\newcommand{\CSLIndent}[1]{\hspace{\cslhangindent}#1}

\begin{document}

% Print the header with above personal informations
% Give optional argument to change alignment(C: center, L: left, R: right)
\makecvheader

% Print the footer with 3 arguments(<left>, <center>, <right>)
% Leave any of these blank if they are not needed
% 2019-02-14 Chris Umphlett - add flexibility to the document name in footer, rather than have it be static Curriculum Vitae
\makecvfooter
  {11 de enero de 2023}
    {Mónica Echeverría Burbano~~~·~~~Hoja de Vida Resumida}
  {\thepage}


%-------------------------------------------------------------------------------
%	CV/RESUME CONTENT
%	Each section is imported separately, open each file in turn to modify content
%------------------------------------------------------------------------------



\hypertarget{acerca-de-muxed}{%
\section{Acerca de mí}\label{acerca-de-muxed}}

Mi trabajo profesional ha estado vinculado con el respeto de los
Derechos Humanos, el acceso a la justicia, la resolución de conflictos y
a la equidad de género, desde la realización de estrategias
comunicativas participativas a favor del cambio social. Investigadora de
medios de comunicación, con experiencia en el análisis del quehacer
periodístico en diferentes contextos, la prevención de las violencias
basadas en género en las universidades y diferentes comunidades. Cuento
con experiencia en la estructuración pedagógica de procesos de
aprendizaje virtuales y presenciales.

\hypertarget{investigaciuxf3n}{%
\section{Investigación}\label{investigaciuxf3n}}

\begin{cvskills}
  \cvskill
    {Líneas de Investigación}
    {Comunicación estratégica • Comunicación con enfoque de derechos humanos • Género • Ciclo de vida • \newline
    Análisis de medios}
\end{cvskills}

\hypertarget{educaciuxf3n}{%
\section{Educación}\label{educaciuxf3n}}

\begin{cventries}
    \cventry{Doctorado en Investigación de Medios de Comunicación}{Universidad Carlos III de Madrid}{Madrid, España}{2022}{}\vspace{-4.0mm}
    \cventry{Maestría en Igualdad de género en ámbito público y privado}{Universidad Jaime I}{Castellón de la Plana, España}{2017}{}\vspace{-4.0mm}
    \cventry{Máster en Derechos Fundamentales}{Universidad Carlos III de Madrid}{Madrid, España}{2011}{}\vspace{-4.0mm}
    \cventry{Especialización en Resolución de Conflictos}{Pontificia Universidad Javeriana}{Bogotá, Colombia}{2009}{}\vspace{-4.0mm}
    \cventry{Comunicación Social}{Universidad del Cauca}{Popayán, Colombia}{2006}{}\vspace{-4.0mm}
\end{cventries}

\hypertarget{experiencia-laboral-y-docente}{%
\section{Experiencia Laboral y
Docente}\label{experiencia-laboral-y-docente}}

Para una lista completa e información detallada, visita mi
\href{https://github.com/JDLeongomez/Moni_HV/raw/main/Monica_Echeverria_HV.pdf}{Hoja
de Vida Académica}.

\begin{cventries}
    \cventry{Docente Investigadora - Programa de Comunicación Social y periodismo}{Universidad Central}{Bogotá, Colombia}{2017 - Actualmente}{\begin{cvitems}
\item Gestión de la comunicación (4 horas semanales - 2018 - Actualmente)
\item Prácticas profesionales (4 horas semanales - 2022 - Actualmente)
\item Gestión de recursos (3 horas semanales - 2021)
\item Opción de grado Comunicación y DDHH - módulo Comunicación y Género (4 horas semanales - 2019 - 2021)
\item Dirección semillero de comunicación y DDHH (2 horas semanales  - 2017 - 2021)
\item Taller de acción social (4 horas semanales - 2017 - 2018)
\item Proyecto de línea - investigación (4 horas semanales - 2018)
\item Prácticas 1 (3 horas semanales - 2017 - 2018)
\end{cvitems}}
    \cventry{Pedagoga}{Cedavida}{Bogotá, Colombia}{Sep 2020 - Dic 2020}{\begin{cvitems}
\item Curso MinTIC por tu mujer
\end{cvitems}}
    \cventry{Coordinadora de Comunicaciones}{Egesco}{Bogotá, Colombia}{Jun 2016 - Dic 2016}{\begin{cvitems}
\item Proyecto “Yo Cuido mi Futuro por Dos” en prevención de embarazo subsiguiente en la adolescencia
\item Creadora de la propuesta
\item Proyecto realizado por el ICBF y Egesco
\end{cvitems}}
    \cventry{Asesora comunicaciones}{Fundación Familia Ayara}{Bogotá, Colombia}{Ago 2015 - Feb 2016}{}\vspace{-4.0mm}
    \cventry{Consultora en Comunicaciones}{Toro Desing Team SAS}{Bogotá, Colombia}{Mar 2015 - Oct 2015}{\begin{cvitems}
\item Realización de la “Sala por la Memoria y la Dignidad de las Víctimas de la Fuerza Pública Colombiana”
\end{cvitems}}
    \cventry{Consultora Experta}{Egesco}{Bogotá, Colombia}{Jun 2014 - Dic 2014}{\begin{cvitems}
\item Proyecto a favor de los Derechos Sexuales y Reproductivos, prevención de las violencias basadas en género e igualdad de género con especial énfasis en prevención de embarazo adolescente para el DPS
\end{cvitems}}
    \cventry{Directora de Comunicaciones}{Checchi and Company Consulting Colombia}{Bogotá, Colombia}{Feb 2013 - Feb 2014}{\begin{cvitems}
\item Proyecto de Acceso a la Justicia
\end{cvitems}}
    \cventry{Comunicadora}{Naciones Unidas - ONU Mujeres}{Pasto, Colombia}{Oct 2011 - May 2012}{\begin{cvitems}
\item Programa Conjunto Ventana de Paz
\end{cvitems}}
    \cventry{Asesora en Comunicaciones}{Oficina de Derechos Humanos de la Vicepresidencia de la República de Colombia}{Bogotá, Colombia}{Nov 2009 - Jul 2010}{\begin{cvitems}
\item Oficina de Lucha Contra la Impunidad
\end{cvitems}}
\end{cventries}

\hypertarget{publicaciones}{%
\section{Publicaciones}\label{publicaciones}}

\hypertarget{section}{%
\subsection{\texorpdfstring{\textbf{Artículos}}{}}\label{section}}

\begingroup
\setlength{\parindent}{-0.5in}
\setlength{\leftskip}{0.5in}

\textbf{Echeverría-Burbano, M.} \& Walzer Moskovic, A. (2022). El
periodismo y las violencias sexuales en el conflicto armado colombiano.
\emph{Revista Opción, 28}, 236-252.
\url{https://www.doi.org/10.5281/zenodo.7278514}

\textbf{Echeverría-Burbano, M.}, \& Leguízamo-Serna, L.R. (2021). El
quehacer periodístico en casos de violencia sexual contra mujeres en
Colombia. \emph{Nómadas, 53}, 268-277.
\url{https://doi.org/10.30578/nomadas.n53a16}

\textbf{Echeverría-Burbano, M.} (2021). El Mito del Amor Romántico en el
relato periodístico sobre los feminicidios de mujeres mayores de edad en
Colombia. \emph{Comunicación Y Sociedad, 18}, 1-19.
\url{https://doi.org/10.32870/cys.v2021.7762}

\endgroup

\hypertarget{section-1}{%
\subsection{\texorpdfstring{\textbf{Capítulos de Libro}}{}}\label{section-1}}

\begingroup
\setlength{\parindent}{-0.5in}
\setlength{\leftskip}{0.5in}

\textbf{Echeverría-Burbano, M.}, \& Leguízamo-Serna, L.R. (2022). El
problema de poner a los victimarios en el centro de las historias de la
violencia en contra de las niñas y las mujeres. El caso del feminicidio
agravado de Yuliana Samboní. En L. Manrique Villanueva \& S.L. Ruiz
Moreno (Eds.), \emph{Mujeres, comunicación y cambio social} (pp.~31-52).
Ediciones USTA. \url{http://hdl.handle.net/11634/44849}

\textbf{Echeverría-Burbano, M.}, \& Baquero Gaitán, M. (2020). El
feminicidio en la prensa escrita colombiana. Un acercamiento desde el
análisis de contenido. En J.M. Pereira G. (Ed.), \emph{Comunicación,
información y lenguajes de la memoria} (pp.~1-17). Editorial Pontificia
Universidad Javeriana.
\url{https://www.javeriana.edu.co/unesco/comunicacioninformacion/contenido/ponencias/tema6/pdf/ponencia_03.pdf}

\textbf{Echeverría-Burbano, M.}, \& Niño Sandoval, Y.P. (2019). Redes
sociales y naturalización de la violencia contra las mujeres. En J.M.
Pereira G. (Ed.), \emph{Buen vivir, cuidado de la casa común y
reconciliación} (pp.~1-15). Editorial Pontificia Universidad Javeriana.
\url{https://javeriana.edu.co/unesco/buenvivir/contenido/ponencias/tema1/pdf/ponencia_15.pdf}

\textbf{Echeverría-Burbano, M.} (2019). Militares de EEUU y abusos
sexuales a niñas en Colombia, un análisis desde los medios colombianos.
En R. Cabral, A.I. Arévalo Salinas, G. Vilar Sastre \& T. Al Najjar
Trujillo (Eds.), \emph{Estudios interdiciplinarios:Paz y comunicación}
(pp.~70-80). Red de universidades internacionales y nacionales.

\endgroup



\end{document}
